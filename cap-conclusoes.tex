%% ------------------------------------------------------------------------- %%
\chapter{Conclusões}
\label{cap:conclusoes}

\section{Objetivas}
\label{sec:objetivas}

    Neste trabalho, foi feito um estudo sobre usabilidade para melhor compreendimento da área, e sobre a avaliação de usabilidade, para definir as melhores maneiras de coletar e analisar os dados neste contexto.
    Após o estudo, os métodos mais apropriados foram escolhidos  e aplicados na avaliação somativa realizada com usuários ativos da rede acadêmica colaborativa Stoa por meio de um questionário, buscando identificar um \emph{ranking} das funcionalidades com maiores problemas de usabilidade na rede. Com esses problemas identificados, foram selecionadas e documentadas as 10 funcionalidades com maiores problemas de usabilidade para serem melhoradas, proporcionando assim uma melhor interação do usuário com a rede.
    O projeto Stoa, por utilizar a plataforma Noosfero e depender de seus lançamentos, não pôde ter as melhorias realizadas inclusas na versão em produção, por isso não foi possível realizar um teste de usabilidade com os usuários para confirmar se as mudanças feitas realmente melhoraram a experiência do usuário com o sistema. Além disso, conforme será explicado na seção seguinte, o questionário continha muitas falhas, que prejudicaram a taxa de respostas e a análise dos dados. Devido a esses problemas encontrados ao longo do trabalho, este estudo ainda será continuado, com um questionário aprimorado e mais focado, conforme será explicado na seção ~\ref{sec:futuro}.
    
\section{Subjetivas}
\label{sec:subjetivas}

    Conforme citado na seção ~\ref{sec:motivos}, os principais interesses em realizar este trabalho eram compreender de forma mais profunda a importância do usuário no desenvolvimento de um produto e colaborar com a disseminação do conhecimento. Com base no estudo bibliográfico realizado e na experiência de analisar na prática a influência do ser humano em uma rede, pode-se dizer que este trabalho possibilitou um ganho pessoal muito significativo, juntamente com a vontade de dar continuidade a ele e aprimorar ainda mais a usabilidade do Stoa.
   
\subsection{Desafios e frustrações}

    A principal dificuldade que eu tive foi a inexperiência de colocar em prática os métodos e métricas para aplicação do questionário \emph{online} dos usuários. Com relação ao tempo, eu subestimei o processo de elaboração do questionário. Com relação à amostra, foi difícil decidir se haveria privilégio para algum grupo de usuários ou não. Além disso, ao formulá-lo, não tive em mente o objetivo principal da pesquisa; foquei primeiro no questionário e depois no tratamento dos dados, quando deveria me valer de como os dados seriam tratados para prepará-lo. Consequentemente, houve diversas perguntas desnecessárias à priorização das funcionalidades, como as relacionadas a informações pessoais dos usuários. Além disso, vários grupos continham apenas uma ou duas funcionalidades, de modo que a comparação destas com outras se tornou impossível com os dados obtidos, e outros grupos abordavam perguntas isoladas, impossibilitando uma análise adequada dos recursos envolvidos.
    
    Outra frustração foi não haver a confirmação das melhorias a tempo da conclusão deste trabalho, devido ao sistema de lançamento a que o Stoa está submetido. As melhorias devem ser feitas e testadas localmente, para depois entrarem em uma lista de espera para serem aprovadas por gestores da plataforma Noosfero, processo que demora meses, principalmente no momento atual, devido a uma migração de versão que está sendo finalizada.

\subsection{Disciplinas cursadas relacionadas}

    Com este trabalho, foi possível unir conhecimentos de diversas áreas, como:
    \begin{itemize}
    \item Usabilidade\\
    A disciplina de Interação-Humano-Computador, oferecida na Karlsruher Institute of Technology (KIT), na Alemanha, me proporcinou um aprendizado teórico, que pôde ser colocada em parte em prática e aprofundada ao longo desta pesquisa;
    \item Estatística\\
    As disciplinas de estatística, oferecidas na graduação do curso de Bacharelado em Ciência da Computação do Instituto de Matemática e Estatística da USP, foram importantes para a decisão de aplicar um índice e para a definição do mesmo, além de terem sido extremamente necessárias para a formulação do questionário e para a escolha da amostra;
    \end{itemize} 


\section{Próximos passos}
\label{sec:futuro} 

    Como continuação desta pesquisa, serão elaborados vídeos tutoriais das principais funcionalidades e tarefas, facilitando a compreensão dos recursos da rede pelo usuário.

    Paralelamente, será aplicado um novo questionário, melhorado a partir dos problemas encontrados após a aplicação do questionário anterior, visando aumentar a quantidade de respostas, por torná-lo mais objetivo e sucinto, e facilitar o tratamento dos dados, por meio do enfoque em apenas duas categorias. Este novo questionário, bem como suas mudanças, será descrito na subseção abaixo. Após aplicação deste questionário e nova priorização dos problemas de usabilidade, será determinado o \emph{ranking} dos maiores problemas de usabilidade.

    Finalmente, enquanto as melhorias forem desenvolvidas, estas serão avaliadas por meio de testes de usabilidade cm grupos pequenos de usuários antes e depois de entrarem em produção, para validar se as mudanças realmente aprimoraram a experiência do usuário.

\subsection{Questionário revisado}

    Devido à baixa tava de respostas obtida pelo questionário tratado neste trabalho e à sua extensão, viu-se a necessidade de formular um novo questionário, mais específico e claro. 

    O público-alvo utilizado no questionário inicial, de usuários ativos e recentes do sistema, foi adequado, pois englobou os principais perfis de usuários do Stoa - professores, funcionários e estudantes- que estavam familiarizados com as funcionalidades mais recentes do sistema, portanto se manteve. A diferença com relação aos participantes é o aumento da amostra para todos os usuários que se logarem no mês anterior em que o novo questionário for aplicado, ao contrário de 1000 usuários na primeira avaliação. Além disso, no tratamento dos dados, será feita uma classificação dos principais problemas de usabilidade por cada perfil citado acima, não só uma classificação geral.

    Para cada pergunta do questionário inicial, foi verificado se ela influenciava no objetivo deste trabalho: a priorização dos principais problemas de usabilidade presentes da rede acadêmica e colaborativa Stoa. Caso a pergunta são influenciasse no objetivo desse estudo, era automaticamente eliminada da seleção para  novo questionário.


   Dessa forma, as seções 1 e 2 inteiras do questionário, "Conhecendo o usuário" e "Verificando o ambiente e tempo de utilização da rede", respectivamente, com exceção da pergunta sobre o vínculo com a USP presente na seção 1, são desnecessárias, pois o enfoque delas é no perfil do indivíduo que utiliza o sistema, contexto que não é avaliado por este trabalho. Como explicado acima, haverá um \emph{ranking} para cada perfil, portanto, o jeito menos invasivo de obter essa informação é apenas manter a pergunta sobre o vínculo com a USP, contendo apenas as alternativas "Estudante", "Professor", "Funcionário" e "Não possuo vínculo atual com a USP".



    A seção 3 do questionário tratava das funcionalidades presentes no sistema, portanto influenciavam no objetivo do sistema. No entanto, esta seção estava mal estruturada, com funcionalidades em excesso, divididas em muitas categorias e com muitas perguntas isoladas. Dessa forma, foi necessário reformular esta seção: 

\begin{itemize}
\item O enfoque nas funcionalidades deixa de ser geral e passa a ser focado em funcionalidades relacionadas a tarefas envolvendo alguma disciplina ou o compartilhamento de conteúdo, os principais objetivos do sistema; 

\item As categorias deixam de ser 5 ("Buscas", "Informações sobre o Stoa Social", "Customização e gerenciamento de Perfil", "Gerenciamento de conteúdos" e "Sobre a visualização das funcionalidades no geral") e passam a ser apenas 2 principais ("Funcionalidades de Perfil de Usuário"  e "Funcionalidades de Comunidades"), que englobam, de alguma maneira, todas as funcionalidades presentes nas categorias anteriores e relacionadas a disciplinas ou ao compartilhamento de conteúdo;

\item Todas as funcionalidades contidas em uma categoria, que antes eram avaliadas por perguntas distintas, passam a ser avaliadas independentemente com a mesma métrica, uma avaliação por parte do usuário de 1 a 5 para cada funcionalidade, de modo que se torna possível compará-las posteriormente para a priorização dos problemas de usabilidade;

\item As funcionalidades deixam de ser avaliadas como macro-funcionalidades, por exemplo, "Gerenciamento de formulários(criação, edição, remoção, visualização de submissões)", e passam a ser avaliadas separadamente, no caso, "Criação de formulário", Edição de formulário", "Remoção de formulário" e "Visualização de submissões do formulário".

\end{itemize}



    A seção 4, por apenas finalizar o questionário e pedir um \emph{feedback} ao usuário, é mantida no questionário novo.


\subsubsection{Estrutura do questionário}

    Aplicando todas as mudanças necessárias, acima citadas, o novo questionário ficará com a seguinte estrutura:

\begin{itemize}
\item {
    Primeira seção:
    \begin{itemize}
    \item  explicação sobre o formulário;
    \item  pergunta sobre o vínculo do usuário com a USP;
    \end{itemize}
}
\item {
    Segunda seção:
    \begin{itemize}
    \item pergunta "Qual das funcionalidades abaixo sobre perfil de Usuário você já utillizou?", com macro-funcionalidades sobre perfil de usuário nas opções
    \item para cada macro-funcionalidade que o participante marcar, serão exibidas novas perguntas, cada uma relativa a uma sub-funcionalidade da macro-funcionalidade, solicitando ao participante que avalie a sub-funcionalidade com uma nota de 1 (muito ruim) a 5 (muito bom)
    \item para as macro-funcionalidades que o participante não marcar, nada será exibido
    \end{itemize}
}
\item {
    Terceira seção:
    \begin{itemize}
    \item pergunta "Qual das funcionalidades abaixo sobre Comunidades você já utillizou?", com macro-funcionalidades soobre comunidades nas opções
    \item para cada macro-funcionalidade que o participante marcar, serão exibidas novas perguntas, cada uma relativa a uma sub-funcionalidade da macro-funcionalidade, solicitando ao participante que avalie a sub-funcionalidade com uma nota de 1 (muito ruim) a 5 (muito bom)
    \item para as macro-funcionalidades que o participante não marcar, nada será exibido
    \end{itemize}
}
\item {
    Quarta seção: será igual à seção de finalização do questionário anterior:
    \begin{itemize}
    \item pergunta se o participante responderia a um novo questionário
    \item em caso afirmativo, o email do usuário é solicitado
    \item pergunta aberta pedindo \emph{feedback} do questionário ao participante
    \end{itemize}
}
\end{itemize}

%todas as funcionalidades listadas como prioritárias serão melhoradas e, conforme forem saindo atualizações do Noosfero, estas %melhorias serão reavaliadas para checagem destas mudanças. Também serão elaborados vídeos tutoriais das principais %funcionalidades e tarefas, facilitando a compreensão dos recursos da rede pelo usuário.

%texto texto texto texto texto texto\footnote{Exemplo de refer�ncia para p�gina
%Web: \url{www.vision.ime.usp.br/~jmena/stuff/tese-exemplo}}.

%------------------------------------------------------
%\section{Considerações Finais} 


%Finalmente, leia o trabalho de Uri Alon %\cite{alon09:how} no qual apresenta-se
%uma reflex�o sobre a utiliza��o da Lei de Pareto para tentar definir/escolher
%problemas para as diferentes fases da vida acad�mica.  A dire��o dos novos
%passos para a continuidade da vida acad�mica deveriam ser discutidos com seu
%orientador.
