%% ------------------------------------------------------------------------- %%
\chapter{Conclusões}
\label{cap:conclusoes}

\section{Objetivas}
\label{sec:objetivas}

    Neste trabalho, foi feito um estudo sobre usabilidade e termos relacionados, para melhor compreendimento da área, e sobre testes envolvendo usuários, para definir as melhores maneiras de coletar e analisar os dados neste contexto.
    Após o estudo, os métodos mais apropriados foram escolhidos e foi então realizada uma avaliação com usuários da rede acadêmica colaborativa Stoa por meio de um formulário, buscando identificar as principais falhas de usabilidade na rede. Com essas questões identificadas, foram selecionadas e documentadas as 10 principais para serem melhoradas, proporcionando assim uma melhor interação do usuário com a rede.
    O projeto Stoa, por utilizar a plataforma Noosfero e depender de seus lançamentos, não pôde ter as melhorias realizadas inclusas na versão em produção, por isso não foi possível realizar um novo teste de usabilidade com os usuários para confirmar se as mudanças feitas realmente melhoraram a experiência do usuário com o sistema, de forma que este estudo ainda será continuado, conforme será explicado na seção ~\ref{sec:futuro}.
    
\section{Subjetivas}
\label{sec:subjetivas}

    Conforme citado na seção ~\ref{sec:motivos}, os principais interesses em realizar este trabalho eram compreender de forma mais profunda a importância do usuário no desenvolvimento de um produto e colaborar com a disseminação do conhecimento. Com base no estudo bibliográfico realizado e na experiência de analisar na prática a influência do ser humano em uma rede, pode-se dizer que este trabalho possibilitou um ganho pessoal muito significante, juntamente com a vontade de dar continuidade a ele.
   
\subsection{Desafios e frustrações}

    A principal dificuldade que eu tive foi a inexperiência de colocar em prática os métodos e métricas para realização do teste de usabilidade dos usuários. Com relação ao tempo, eu subestimei o processo de elaboração do questionário. Com relação à amostra, foi difícil decidir se haveria privilégio para algum grupo de usuários ou não. Além disso, ao formulá-lo, não tive em mente o objetivo principal da pesquisa; foquei primeiro no formulário e depois no tratamento dos dados, quando deveria me valer de como os dados seriam tratados para prepará-lo. Consequentemente, houve diversas perguntas desnecessárias à priorização das funcionalidades, como as relacionadas a informações pessoais dos usuários. Além disso, vários grupos possuíram apenas uma ou duas funcionalidades, de modo que a comparação destas com outras se tornou impossível com os dados obtidos, e outros grupos abordavam perguntas isoladas, impossibilitando uma análise adequada dos recursos envolvidos.
    
    Outra frustração foi não haver a confirmação das melhorias a tempo da conclusão deste trabalho, devido ao sistema de lançamento a que o Stoa está submetido. As melhorias devem ser feitas e testadas localmente, para depois entrarem em uma lista de espera para serem aprovadas por gestores da plataforma Noosfero, processo que demora meses, principalmente no momento atual, devido a uma migração de versão que está sendo finalizada.

\subsection{Disciplinas cursadas relacionadas}

    Com este trabalho, foi possível unir conhecimentos de diversas áreas. São eles:
    \begin{itemize}
    \item Usabilidade\\
    A matéria de Interação-Humano-Computador, que eu havia aprendido na teoria, pôde ser colocada em prática e aprofundada ao longo desta pesquisa;
    \item Estatística\\
    As matérias de estatísticas que vemos na graduação foram importantes para a decisão de aplicar um índice e para a definição do mesmo, além de terem sido extremamente necessárias para a formulação do questionário e para a escolha da amostra;
    \item Engenharia de Software\\
    As matérias Programação Extrema, Engenharia de Software e Programação Orientada a Objetos foram essenciais para se trabalhar corretamente em um projeto grande, delegando funções quando necessário, mantendo as atividades atualizadas, entre outras contribuições.
    \end{itemize} 
\section{Próximos passos}
\label{sec:futuro} 

    Como continuação desta pesquisa, todas as funcionalidades listadas como prioritárias serão melhoradas e, conforme forem saindo atualizações do Noosfero, estas melhorias serão reavaliadas para checagem destas mudanças. Também serão elaborados vídeos tutoriais das principais funcionalidades e tarefas, facilitando a compreensão dos recursos da rede pelo usuário.

%texto texto texto texto texto texto\footnote{Exemplo de refer�ncia para p�gina
%Web: \url{www.vision.ime.usp.br/~jmena/stuff/tese-exemplo}}.

%------------------------------------------------------
%\section{Considerações Finais} 


%Finalmente, leia o trabalho de Uri Alon %\cite{alon09:how} no qual apresenta-se
%uma reflex�o sobre a utiliza��o da Lei de Pareto para tentar definir/escolher
%problemas para as diferentes fases da vida acad�mica.  A dire��o dos novos
%passos para a continuidade da vida acad�mica deveriam ser discutidos com seu
%orientador.
